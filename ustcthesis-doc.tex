% !TeX encoding = UTF-8
% !TeX program = xelatex
% !TeX spellcheck = en_US

%
% Copyright (C) 2015-2025 by USTC TeX Users Group <https://github.com/ustctug>
%
% This file may be distributed and/or modified under the
% conditions of the LaTeX Project Public License, either version 1.3c
% of this license or (at your option) any later version.
% The latest version of this license is in
%    https://www.latex-project.org/lppl.txt
% and version 1.3c or later is part of all distributions of LaTeX
% version 2008 or later.
%

\documentclass[a4paper]{ltxdoc}

\usepackage[UTF8]{ctex}
\usepackage{unicode-math}
\usepackage{caption}
\usepackage{booktabs}
\usepackage{xcolor}
\usepackage{listings}
\usepackage[perpage]{footmisc}
\usepackage{hypdoc}

\makeatletter

% 设置字体
\IfFileExists{/System/Library/Fonts/Times.ttc}{
  \setmainfont{Times}
  \setsansfont[Scale=MatchLowercase]{Helvetica}
  \setmonofont[Scale=MatchLowercase]{Menlo}
  \xeCJKsetwidth{‘’“”}{1em}
}{}
\unimathsetup{
  math-style=ISO,
  bold-style=ISO,
}
\IfFontExistsTF{xits-math.otf}{
  \setmathfont[
    Extension    = .otf,
    BoldFont     = *bold,
    StylisticSet = 8,
  ]{xits-math}
  \setmathfont[range={cal,bfcal},StylisticSet=1]{xits-math.otf}
}{
  \setmathfont[
    Extension    = .otf,
    BoldFont     = XITSMath-Bold,
    StylisticSet = 8,
  ]{XITSMath-Regular}
  \setmathfont[range={cal,bfcal},StylisticSet=1]{XITSMath-Regular.otf}
}

% 定义一些命令用于写文档
\newcommand\TeXLive{\TeX{} Live}
\newcommand\unicodechar[1]{U+#1(\symbol{"#1})}
\DeclareRobustCommand\file{\nolinkurl}
\DeclareRobustCommand\env{\texttt}
\DeclareRobustCommand\pkg{\textsf}
\DeclareRobustCommand\cls{\textsf}
\DeclareRobustCommand\opt{\texttt}

% 在 doc 的基础上增加 option 的描述
\def\DescribeOption{\leavevmode\@bsphack\begingroup\MakePrivateLetters
  \Describe@Option}
\def\Describe@Option#1{\endgroup
              \marginpar{\raggedleft\PrintDescribeOption{#1}}%
              \SpecialEnvIndex{#1}\@esphack\ignorespaces}
\@ifundefined{PrintDescribeOption}
   {\def\PrintDescribeOption#1{\strut \MacroFont #1\ }}{}

% 调整列表的格式
\setlength\partopsep{\z@}
\def\@listi{\leftmargin\leftmargini
            \parsep \z@
            \topsep \z@
            \itemsep \z@}
\let\@listI\@listi
\@listi

% listings 的样式
\lstdefinestyle{lstshell}{
  basicstyle      = \small\ttfamily,
  backgroundcolor = \color{lightgray},
  language        = bash,
}
\newcommand\shellcmd[1]{\colorbox{lightgray}{\lstinline[style=lstshell]|#1|}}
\lstnewenvironment{shell}{\lstset{style=lstshell, gobble=2}}{}
\lstnewenvironment{latex}{%
  \lstset{
    basicstyle = \small\ttfamily,
    frame      = single,
    gobble     = 2,
    language   = [LaTeX]TeX,
  }%
}{}

\hypersetup{
  allcolors         = blue,
  bookmarksnumbered = true,
  bookmarksopen     = true,
}
\makeatother


\begin{document}



\title{\cls{ustcthesis} 使用说明}
\author{Zeping Lee\thanks{zepinglee AT gmail.com} \and
        seisman\thanks{seisman.info AT gmail.com} }
\date{v4.0.0-beta.6\qquad 2025-02-25}
\maketitle



\section{简介}

本模板 \cls{ustcthesis} 是中国科学技术大学本科生和研究生学位论文的 \LaTeX{}
模板, 按照《中国科学技术大学研究生学位论文撰写手册》(最近在修订中)和
《\href{https://www.teach.ustc.edu.cn/?attachment_id=13867}
{中国科学技术大学本科毕业论文(设计)格式}》的要求编写。

其前身是中国科学技术大学本科论文模板(作者 XPS,最后维护 ywg)
和中国科学技术大学研究生论文模板(作者 Liuqs,主要维护 Liuqs、Guolicai)。
后来两模板进行了整合梳理,由 ywg 维护。
2015 年,seisman 和 zepinglee 基于 \pkg{ctex} 2.0 重新编写了模板。
2017 年,随着学校发布了新版的《撰写手册》,本模板也更新到 v3.0。

下载地址:
\begin{itemize}
  \item 主要地址:\url{https://github.com/ustctug/ustcthesis/releases}
  \item 学校镜像:\url{https://git.lug.ustc.edu.cn/ustctug/ustcthesis}
  \item 研究生院网站(版本较旧):
    \url{https://gradschool.ustc.edu.cn/column/65}
\end{itemize}

用户在使用 \pkg{ustcthesis} 模板前,应先阅读学校的《撰写手册》等规范。
如果在使用的过程中遇到问题,可以阅读
\href{https://github.com/ustctug/ustcthesis/wiki}{常见问题},
或者在 \href{https://github.com/ustctug/ustcthesis/issues}{GitHub Issues}
中反馈。



\section{编译方法}


\subsection{文件组成}
本模板的主要文件如表~\ref{tab:files}:
\begin{table}[htb]
  \centering\small
  \caption{模板的文件组成}
  \label{tab:files}
  \begin{tabular}{lll}
    \toprule
    类别     & 文件                      & 说明                         \\
    \midrule
    模板文件 & \file{ustcthesis.cls}     & 模板文件                     \\
             & \file{ustcthesis-*.bst}   & \BibTeX{} 参考文献表格式     \\
             & \file{ustcthesis-*.bbx}   & BibLaTeX 参考文献表格式      \\
             & \file{ustcthesis-*.cbx}   & BibLaTeX 参考文献表格式      \\
             & \file{figures/ustc-*.pdf} & 校名和校徽图片               \\
    \midrule
    使用说明 & \file{ustcthesis-doc.tex} & 模板使用说明的源代码         \\
             & \file{ustcthesis-doc.pdf} & (你正在阅读的)模板使用说明 \\
    \midrule
    示例文档 & \file{main.tex}           & 主文档                       \\
             & \file{ustcsetup.tex}      & 配置文件                     \\
             & \file{chapters/*.tex}     & 示例文档的各个章节           \\
             & \file{figures/}           & 放置图片的目录               \\
             & \file{bib/ustc.bib}       & \BibTeX{} 示例数据库         \\
    \midrule
    其他     & \file{README.md}          & 基本说明                     \\
             & \file{CHANGELOG.md}       & 更新日志                     \\
             & \file{latexmkrc}          & latexmk 的配置文件           \\
             & \file{Makefile}           & GNU make 的配置文件          \\
             & \file{.vscode/}           & VS Code 的配置文件           \\
             & \file{build.lua}          & l3build 的配置文件           \\
             & \file{test}               & l3build 的测试文件           \\
    \bottomrule
  \end{tabular}
\end{table}

示例文档包括了常用的 \LaTeX{} 命令,建议新手从此入手,用自己的内容进行替换。


\subsection{依赖宏包}

本模板要求使用 TeX Live、MacTeX 或 MiKTeX 不低于 2017 年的发行版,
推荐升级到最新的版本。

模板直接依赖的宏包有:
\pkg{amsmath},
\pkg{caption},
\pkg{color},
\pkg{ctex},
\pkg{fancyhdr},
\pkg{footmisc},
\pkg{geometry},
\pkg{graphicx},
\pkg{notoccite},
\pkg{titletoc},
\pkg{url},
\pkg{unicode-math}。

另外,模板还对其他宏包提供了支持,包括:
\pkg{amsthm},
\pkg{algorithm2e},
\pkg{natbib},
\pkg{biblatex},
\pkg{hyperref},
\pkg{nomencl},
\pkg{siunitx}。
这些宏包并非必需,用户可以根据需要选择使用。
模板在检测到这些宏包被调用后会自动进行配置。

注意,本模板\emph{不}兼容的宏包有:
\pkg{amsfonts},
\pkg{amssymb},
\pkg{bm},
\pkg{cite},
\pkg{mathrsfs},
\pkg{newtx},
\pkg{upgreek}。


\subsection{开始编译}

\begin{enumerate}

\item GNU make \\
Linux/Mac用户,可以直接使用 GNU make 工具,这是最简单的方法。
编译论文 \file{main.pdf}:
\begin{shell}
  make
\end{shell}
编译说明文档 \file{ustcthesis-doc.pdf}:
\begin{shell}
  make doc
\end{shell}
另外还可以用 \shellcmd{make clean} 清理辅助文件。

\item |latexmk| \\
Windows 用户可能无法使用GNU make,使用 |latexmk| 也是一个比较简单的方法,
配置文件由 \file{latexmkrc} 给出,其参数设置为 |-xelatex|,用户编译论文
只需使用命令:
\begin{shell}
  latexmk -xelatex main.tex
\end{shell}
编译说明文档:
\begin{shell}
  latexmk -xelatex ustcthesis-doc.tex
\end{shell}
清理辅助文件可以用 \shellcmd{latexmk -c}。图形界面用户应参考编辑器的使用说明。

\item 手动编译 \\
手动编译是最繁琐的方法,用户可能需要运行多遍,以确保论文的交叉引用等信息全部正确。

编译论文 \file{main.pdf}:
\begin{shell}
  xelatex main.tex
  bibtex main.aux
  xelatex main.tex
  xelatex main.tex
\end{shell}
编译说明文档 \file{ustcthesis-doc.pdf}:
\begin{shell}
  xelatex ustcthesis-doc.tex
  xelatex ustcthesis-doc.tex
\end{shell}
\end{enumerate}



\section{模板设置}


\subsection{文档类参数}
\DescribeOption{degree}
选择学位,支持 \opt{bachelor},\opt{master},\opt{doctor}(默认)。
\begin{latex}
  \documentclass[degree=doctor]{ustcthesis}
\end{latex}

\DescribeOption{degree-type}
学位类型。可选:学术型 \opt{academic}(默认),专业型 \opt{professional}。
\begin{latex}
  \documentclass[degree-type=professional]{ustcthesis}
\end{latex}

\DescribeOption{language}
论文全文的主要语言。可选:\opt{chinese}(默认),\opt{english}。
\begin{latex}
  \documentclass[language=english]{ustcthesis}
\end{latex}

\DescribeOption{output}
输出 PDF 的类型:
\begin{itemize}
  \item \opt{print}(默认):用于双面打印纸质论文
  \item \opt{electronic}:单面,并保留超链接颜色
\end{itemize}
\begin{latex}
  \documentclass[output=electronic]{ustcthesis}
\end{latex}

\DescribeOption{section-style}
本科生专用,章节标题的样式。可选:\opt{chinese}(默认),\opt{arabic}。
\begin{itemize}
  \item \opt{chinese}(默认):汉字序号
  \item \opt{arabic}:阿拉伯数字序号
\end{itemize}
\begin{latex}
  \documentclass[section-style=arabic]{ustcthesis}
\end{latex}


\subsection{字体设置}
模板默认会自动根据操作系统配置合适的字体,
\DescribeOption{fontset}
用户也可以通过 \opt{fontset} 时指定使用预设的字库,如:
\begin{latex}
  \documentclass[fontset=windows]{ustcthesis}
\end{latex}
允许的选项有 \opt{windows}、\opt{mac}、\opt{ubuntu} 和 \opt{fandol},
具体使用的字体见表~\ref{tab:fontset}。
用户也可以设置为 \opt{none} 并自行配置字体。

\begin{table}[htb]
  \centering
  \caption{模板预设的字体}
  \label{tab:fontset}
  \begin{tabular}{cccc}
    \toprule
    \opt{windows} & \opt{mac}    & \opt{ubuntu} & \opt{fandol} \\
    \midrule
    Times New Roman  & Times New Roman & TeX Gyre Termes & TeX Gyre Termes \\
    Arial            & Arial           & TeX Gyre Heros  & TeX Gyre Heros  \\
    Courier          & Menlo           & TeX Gyre Cursor & TeX Gyre Cursor \\
    中易宋体         & 华文宋体        & 思源宋体        & Fandol 宋体     \\
    中易黑体         & 华文黑体        & 思源黑体        & Fandol 黑体     \\
    \bottomrule
  \end{tabular}
\end{table}

然而自动配置的字体只能保证编译通过,但是还存在一些问题:
\begin{enumerate}
  \item 在其他平台上配置的 TeX Gyre 系列字体,虽然在风格上比较接近 Times 和 Arial,
    但是毕竟跟《撰写手册》要求的字体不完全一致;
  \item Fandol 字库的字形较少,常常出现缺字的情况;
  \item 华文字库和 Fandol 字库虽然不违反《撰写手册》的要求,
    但是其字形跟中易字库有所差别,可能被审查老师认为格式不符合要求。
\end{enumerate}

所以建议在提交最终版前使用 Windows 平台的字体进行编译。



\section{论文内容}

\subsection{封面与声明}

“封面”的名字让人有些混淆,它既可以指由印刷厂统一制作的硬皮封面(cover),也可
以指书打开后的第一页(title page)。在这里指的是后者,所以本模板从 title page
开始。

\DescribeMacro{\maketitle}
封面由 \cs{maketitle} 命令生成,其中的各项信息使用 \cs{ustcsetup} 命令以
key-value 的方式填写,如:
\begin{latex}
  \ustcsetup{
    title  = {论文中文题目},
    title* = {Thesis English Title},
  }
\end{latex}
模板提供的选项见表~\ref{tab:covercmds},
\begin{table}[htb]
  \centering\small
  \caption{录入封面信息的命令}
  \label{tab:covercmds}
  \begin{tabular}{lll}
    \toprule
    命令               & 命令(英文)        & 说明          \\
    \midrule
    \opt{title}        & \opt{title*}        & 论文标题      \\
    \opt{author}       & \opt{author*}       & 作者姓名      \\
    \opt{speciality}   & \opt{speciality*}   & 学科专业      \\
    \opt{supervisor}   & \opt{supervisor*}   & 导师      \\
    \opt{supepractice-supervisorrvisor} & \opt{practice-supervisor*} & 导师 \\
    \opt{date}         & -                   & 完成时间      \\
    \opt{secret-level} & \opt{secret-level*} & 密级          \\
    \opt{secret-year}  & -                   & 保密/控阅年限 \\
    \bottomrule
  \end{tabular}
\end{table}

有几点说明:
\begin{itemize}
  \item \cs{ustcsetup} 使用 \pkg{kvsetkeys} 机制,配置项之间不能有空行,否则会报错。
  \item 其中带 |*| 后缀的选项用于设置英文封面。
  \item 导师姓名 \opt{supervisor} 允许多个姓名,使用“,”(西文逗号 U+002C)隔开。
  \item 完成时间 \opt{date} 应使用 ISO 格式(yyyy-mm-dd),默认为当前日期。
\end{itemize}

\DescribeMacro{\copyrightpage}
声明页由 \cs{copyrightpage} 命令生成。模板会根据 \opt{secret-level} 和
\opt{secret-year} 的内容自动填写。

\DescribeOption{reviewer}
有的院系额外要求声明页有“评审专家签名”,可以在模板中使用 \cs{ustcsetup} 命令设置
\opt{reviewer = true}。


\subsection{摘要和章节}
\DescribeEnv{abstract}
\DescribeEnv{abstract*}
\DescribeEnv{notation}
\DescribeEnv{acknowledgements}
\DescribeEnv{achievements}
对于特殊的章节,\cls{ustcthesis} 还提供了相应的环境:
\begin{itemize}
  \item 中文摘要:\env{abstract}
  \item 英文摘要:\env{abstract*}
  \item 符号说明:\env{notation}
  \item 致谢:    \env{acknowledgements}
  \item 科研成果:\env{achievements}
  \item 科研成果列表:\env{theachievements}
\end{itemize}

\DescribeOption{keywords}
\DescribeOption{keywords*}
摘要的关键词应使用 \cs{ustcsetup} 的接口进行设置,在录入时使用西文逗号隔开,
比如:
\begin{latex}
  \begin{abstract}
    这里是摘要。
    \ustcsetup{
      keywords  = {学位论文, 摘要, 关键词},
      keywords* = {Dissertation, Abstract, Keywords},
    }
  \end{abstract}
\end{latex}

\DescribeMacro{\tableofcontents}
\DescribeMacro{\listoffigures}
\DescribeMacro{\listoftables}
\DescribeMacro{\listoffiguresandtables}
目录、插图和附表清单可以使用命令自动生成:
\begin{itemize}
  \item 目录:  \cs{tableofcontents}
  \item 插图清单:\cs{listoffigures}
  \item 附表清单:\cs{listoftables}
  \item 插图和附表清单:\cs{listoffiguresandtables}
\end{itemize}


\subsection{插图和表格}

《撰写手册》要求图题置于图的下方,表题置于表的上方。
\LaTeX{} 的 \cs{caption} 命令并不能控制标题的位置,
需要作者注意写在合适的地方。

\DescribeMacro{\figurenote}
本模板还提供了 \cs{note}\marg{notes} 命令,用于在图表中添加注释。

关于图片的并排,推荐使用较新的 \pkg{subcaption} 宏包,不建议使用
\pkg{subfigure} 或 \pkg{subfig}。

更多的表格样式参见 \pkg{booktabs}(三线表)、\pkg{longtable}(跨页表格)。

算法可以使用 \pkg{algorithms} 宏包或者较新的 \pkg{algorithm2e}。


\subsection{数学}

《撰写手册》要求数学符号遵循 GB/T 3102.11—1993《物理科学和技术中使用的数学符号》
\footnote{原 GB 3102.11—1993,自 2017 年 3 月 23 日起,该标准转为推荐性标准。}。
该标准参照采纳 ISO 31-11:1992 \footnote{目前已更新为 ISO 80000-2:2019。},
但是与 \TeX{} 默认的美国数学学会(AMS)的习惯有许多差异。
这将在下文详细论述。

\DescribeOption{math-style}
用户可以通过设置 \opt{math-style} 选择数学符号样式(可选:
\opt{GB}(中文默认),\opt{TeX}(英文默认)和 \opt{ISO}),比如:
\begin{latex}
  \ustcsetup{
    math-style = ISO,
  }
\end{latex}

用户也可以逐项修改数学样式。
\newcommand\dif{\mathop{}\!\mathrm{d}}
\begin{enumerate}
  \item \DescribeOption{uppercase-greek}
    大写希腊字母的正/斜体,可选:\opt{italic}、\opt{upright}。
    有限增量符号 $\increment x$ 固定使用正体,推荐使用 \cs{increment} 表示。
  \item \DescribeOption{less-than-or-equal}
    小于等于号和大于等于号的字形,可选:\opt{slanted}、\opt{horizontal}。
    这将控制 \cs{le}、\cs{ge}、\cs{leq} 和 \cs{geq} 的符号
    是“$\leqslant$、$\geqslant$”还是“$\leq$、$\geq$”。
  \item \DescribeOption{integral}
    积分号的正/斜体,可选:\opt{upright}、\opt{slanted}。
    该选项需要字体的支持,目前仅限 \opt{xits}、\opt{stix}、
    \opt{libertinus} 和 \opt{newcm}。参考下文关于数学字体的选择。
  \item \DescribeOption{integral-limits}
    积分号上下限的位置,可选:\opt{true}(在上下)、\opt{false}(在右侧)。
    这个设置只影响行间公式,行内公式统一居右侧,不受影响。
  \item \DescribeOption{partial}
    偏微分符号的正/斜体,可选:\opt{upright}、\opt{slanted}。
  \item \DescribeOption{math-ellipsis}
    省略号 \cs{dots} 的样式,可选:\opt{centered}(按照中文的习惯固定居中)、
    \opt{lower} 和 \opt{AMS}(取决于前后符号的位置)。
    其他的省略号命令如 \cs{lots}、\cs{cdots} 则不受影响。
  \item \DescribeOption{real-part}
    实部 \cs{Re} 和虚部 \cs{Im} 的字体,可选:\opt{roman} 和 \opt{fraktur}。
\end{enumerate}

如果数学符号选择国标样式 |math-style = GB|,相当于设置了
\begin{latex}
  \ustcsetup{
    uppercase-greek    = italic,
    less-than-or-equal = slanted,
    integral           = upright,
    integral-limits    = false,
    partial            = upright,
    math-ellipsis      = centered,
    real-part          = roman,
  }
\end{latex}

另外,国标的数学样式与 AMS 还有些差异,但无法在模板中统一设置。用户需要在写作时进行处理。
\begin{enumerate}
  \item 数学常数和特殊函数名用正体,如 $\uppi = 3.14\dots$;$\symup{i}^2 = -1$;
    $\symup{e} = \lim_{n \to \infty} \left( 1 + \frac{1}{n} \right)^n$。
  \item 微分号使用正体,比如 $\dif y / \dif x$。
  \item 向量、矩阵和张量用粗斜体(\cs{symbf}),如 $\symbf{x}$、$\symbf{\Sigma}$、$\symbfsf{T}$。
  \item 自然对数用 $\ln x$ 不用 $\log x$。
\end{enumerate}

需要注意,上述关于数学符号风格的设置在设置数学字体(\opt{math-font})时才会生效。

\DescribeOption{math-font}
模板使用默认使用 XITS Math 作为数学字体。
用户也可以使用 \opt{math-font} 选项切换其他数学字体,可选:
\opt{stix}(STIX Two Math)、
\opt{libertinus}(Libertinus Math)、
\opt{newcm}(New Computer Modern Math)、
\opt{lm}(Latin Modern Math)。

其中 \opt{lm} 和 \opt{newcm} 的字形比较搭配 TeX 原生的 Computer Modern 字体,
但与《指南》要求的西文字体 Times New Roman 并不搭配。
可能会造成正文和公式中的数字字体不一致,需要谨慎使用。

以上字体都是 OpenType 格式的字体,需要配合
\href{http://mirrors.ctan.org/macros/latex/contrib/unicode-math/unicode-math.pdf}{\pkg{unicode-math}}
宏包使用。
全部数学符号的命令参考
\href{http://mirrors.ctan.org/macros/latex/contrib/unicode-math/unimath-symbols.pdf}{\pkg{unimath-symbols}}。
注意,\pkg{unicode-math} 宏包与 \pkg{amsfonts}、\pkg{amssymb}、\pkg{bm}、
\pkg{mathrsfs}、\pkg{upgreek} 等宏包\emph{不}兼容。
模板作了处理,用户可以直接使用这些宏包的命令,如 \cs{bm}、\cs{mathscr}、
\cs{uppi}。

另外,模板还为 \opt{math-font} 提供了传统的 Type 1 字体 \opt{newtx}。
该选项会调用 \pkg{newtxmath} 宏包。
但是,如果西文字体已经使用了 OpenType 的 Times New Roman,
混用 Type 1 字体可能会导致问题,尤其是使用 \pkg{siunitx} 宏包时。
该选项还处于测试阶段,需要谨慎使用。

\subsection{参考文献}

参考文献支持使用 \BibTeX{} 或 BibLaTeX 生成。
用户可以根据需求选择合适的方式。


\subsubsection{\BibTeX}

\BibTeX{} 是 \LaTeX{} 处理参考文献的传统的方式。
在本模板中需要载入 \pkg{natbib} 宏包,
然后选择“顺序编码制”或“著者-出版年制”两种格式之一:
\begin{latex}
  % 顺序编码制
  \usepackage[sort]{natbib}
  \bibliographystyle{ustcthesis-numeric}
\end{latex}
或者
\begin{latex}
  % 著者-出版年制
  \usepackage{natbib}
  \bibliographystyle{ustcthesis-authoryear}
\end{latex}
并且在生成参考文献表的位置使用 \cs{bibliography} 设置 \file{.bib} 数据库的路径。
然后使用 \file{bibtex} 编译生成 \file{.bbl} 文件。
这里使用参考文献样式(\file{.bst} 文件)是在
\href{https://ctan.org/pkg/gbt7714}{\pkg{gbt7714}}
的基础上进行了少量修改。

顺序编码制的文内引用标注分为两种样式:
\begin{enumerate}
  \item 上标样式(\opt{super}), 如“张教授\textsuperscript{[2-3]}认为……
    形成了多种物理模型\textsuperscript{[7,9,11-13]}”。
  \item 与正文平排(\opt{inline}),文献序号作为叙述文字的一部分,
    如“张教授\textsuperscript{[4]}认为……根据文献[4]\textsuperscript{101-105}”。
\end{enumerate}

模板默认为上标样式,也可以将引用标注统一设为 \opt{inline} 样式。
\begin{latex}
  \ustcsetup{
    cite-style = inline,
  }
\end{latex}
或者使用 \cs{inlinecite} 在局部与正文平排。

著者-出版年制的文献引用标注也有两种主要模式:
\begin{enumerate}
  \item \cs{citet}:著者姓名作为正文的一部分,比如“Zhang (2008)”,
    同默认的 \cs{cite} 命令一样。
  \item \cs{citep}:著者与年份均在括号中,比如“(Zhang, 2008)”。
\end{enumerate}

参考文献表采用著者-出版年制组织时,各篇文献首先按文种集中, 然后按著者字顺和出版年排列;
中文文献可以按著者汉语拼音字顺排列,也可以按著者的笔画笔顺排列。
但由于 \BibTeX{} 功能的局限性,无法自动获取著者姓名的拼音或笔画笔顺进行正确排序。
一种解决方法是在 \file{.bib} 数据库的中文文献的 |key| 域手动录入著者姓名的拼音:
\begin{latex}
  @book{capital,
    author = {马克思 and 恩格斯},
    key    = {ma3 ke4 si1 & en1 ge2 si1},
    ...
  }
\end{latex}
如果中文文献数量较多,推荐使用 BibLaTeX 处理参考文献。

\paragraph{与 \pkg{chapterbib} 的兼容性}
如果在每一章添加独立的参考文献,可以使用 \pkg{chapterbib} 宏包。
需要注意以下几点:
\begin{enumerate}
  \item \pkg{chapterbib} 宏包与 \pkg{natbib} 的载入顺序不重要。
  \item 根据 \pkg{natbib} 的文档,\opt{sectionbib} 参数(将参考文献的标题改为节标题)
    需要传给 \pkg{natbib} 而不是 \pkg{chapterbib}。
    模板作了设置,如载入 \pkg{chapterbib},\pkg{natbib} 就会有 \opt{sectionbib} 的效果。
  \item 每章的内容需要写在单独的 \file{tex} 文件,并且在主文档中使用 \cs{include},
    不带 \file{.tex} 后缀。
  \item 每章都要使用 \cs{bibliographystyle} 和 \cs{bibliography} 命令。
\end{enumerate}


\subsubsection{BibLaTeX}

BibLaTeX 是较新的参考文献处理方案。
需要在载入宏包时通过选择样式(\opt{style}),
并且在导言区使用 \cs{addbibresource} 设置数据库的路径,
最后在参考文献表的位置使用 \cs{printbibliography} 输出。
\begin{latex}
  \usepackage[style=ustcthesis-authoryear]{biblatex}
  \addbibresource{bib/ustc.bib}
\end{latex}
使用 BibLaTeX 必须用 \file{biber} 编译生成参考文献表。
这里的样式是在 \href{https://ctan.org/pkg/biblatex-gb7714-2015}{biblatex-gb7714-2015}
的基础上进行了少量修改。
一些额外用法可以参考该宏包的文档。

对于顺序编码制,文献引用标注默认为上标样式。
如果要切换为与正文平排,一种方法是使用 \cs{parencite},
另一种是载入 \pkg{biblatex} 时统一设置。
\begin{latex}
  \usepackage[bibstyle=ustcthesis-numeric,
              citestyle=ustcthesis-inline]{biblatex}
\end{latex}



\end{document}
