% !TeX root = ../main.tex

\chapter{总结与展望}

\section{工作总结}

% 本文主要进行以下几个方面的工作:
% 1.首先介绍了课题背景和国内外研究现状,说明了安卓的图形栈对于整个安卓系统的重要性,并且安卓作为一个商业化成熟的项目,优秀的
% 厂家的硬件设计标准和驱动往往闭源,信息的闭塞导致对研究造成一定的壁垒,说明了在目前的安卓框架下实现龙芯图形系统方案的必要性。
% 围绕安卓图形系统介绍了相关技术,包括AOSP项目,图形系统的结构以及GPU的驱动栈,这些构建安卓图形系统的基础

% 2.在安卓5.10内核上实现了龙芯内核模块。分析了现代操作系统对硬件资源管理的设计模式,以DRM框架为核心设计了一套IOCTL系统。
% 并且为了处理内核版本变化带来的接口不一致的问题,设计了一个多版本适配的方案。
% 同时,实现了安卓的两个组件硬件混合渲染模块和图形缓存分配模块。其中图形缓存分配模块设计了缓冲复用模块,并且支持修饰等实现
% 对特殊布局tile和压缩等格式。

% 3.

本文围绕龙芯平台下的安卓图形系统构建,针对国产硬件生态建设的迫切需求,提出并实现了一套完整的图形栈解决方案。主要工作包括:

1.架构设计与技术分析:系统梳理了安卓图形系统的多层次架构,结合龙芯LG110显卡特性,明确了内核驱动、用户态驱动、硬件抽象层(HAL)及系统镜像定制四大核心模块的技术路径。

2.内核驱动开发与适配:基于DRM框架设计了17类IOCTL命令,实现了显存管理、命令流控制等核心功能,并通过多版本适配方案解决了内核接口动态兼容性问题,成功在安卓5.10内核中集成龙芯显卡驱动。

3.用户态驱动与HAL层实现:通过Mesa框架移植OpenGL ES支持,重构了gralloc与HWC模块,首次在龙芯平台上实现Android 12的硬件抽象层兼容,支持图形缓冲区的显存分配、跨进程共享及硬件加速合成。

4.系统级验证与优化:针对龙芯16KB页表特性,定制了安卓内核与Bionic库,完成从内核到应用层的全栈适配,并通过谷歌官方测试集验证了功能正确性,覆盖SurfaceFlinger、渲染引擎等关键组件。

本研究的创新性体现在:首次将安卓图形系统完整移植至龙芯LoongArch架构,填补了国产移动端操作系统生态的空白;通过重构gralloc与HWC模块,实现了国产GPU与安卓框架的高效协同,为后续国产化替代奠定了技术基础。

\section{未来展望}

尽管本研究取得了一定成果,但仍存在以下问题需进一步探索:

1.兼容性与性能优化:当前方案仅适配Android S版本与龙芯3A5000平台,未来需扩展至更高版本安卓(如Android 15)及新一代龙芯硬件(如LG200),并针对Vulkan API支持、多显示器输出等场景进行性能优化。

2.闭源依赖与可维护性:龙芯GPU内核驱动仍为闭源实现,导致驱动升级与问题排查依赖厂商支持。后续可推动开源社区协作,或基于Rust等内存安全语言重构驱动模块,提升可维护性。

3.实际场景验证不足:目前测试集中于功能正确性,缺乏高负载场景(如3D游戏、4K视频渲染)下的性能评估。需构建真实应用测试环境,量化帧率、功耗等关键指标。

4.安全性与生态扩展:未涉及可信执行环境(TEE)与图形数据加密,未来可结合国产密码算法增强图形栈安全性;同时需推动应用生态适配,吸引开发者基于龙芯平台优化图形应用。
