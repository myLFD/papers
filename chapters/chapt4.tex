% !TeX root = ../main.tex

\chapter{性能分析与优化(安全性与稳定性分析)}

\section{测试环境搭建}
由于AOSP仅源码就有130G之多,在完成编译之后达到212G,因此构建Android镜像需要一台构建机以及测试机。
构建机需要充足的存储空间以及编译大量内容,所以选择核数多性能强的服务器,测试机是目标系统镜像的运行平台。

构建机:在构建安卓内核、镜像,以及编译loonggpu核模块都使用的是龙芯公司的X86服务器(所以所有项目都是交叉编译),Xeon Gold 6388处理器,核数
为128核。

构建环境配置:
    AOSP镜像构建:AOSP配置环境使用预先编译好的llvm15的编译器(这些前期已完成),通过AOSP项目提供的envsetup.sh脚本设置基本的环境变量,并lunch loongarch-3A5000选择构建目标,
        设置特有的环境变量。编译上可以选择使用m编译全平台代码,mm编译某模块代码及其依赖或者使用mmm编译某个指定路径下的所有路径及其依赖。
    Android内核构建:编译工具链使用loongson\ gnu\ toolchain 8.3,使用自写脚本设置交叉编译环境变量,主要是PATH、LD\_LIBRARY\_PATH、CROSS\_COMPILE、ARCH等变量的指定,
        最后使用make编译,可以使用-j(\$proc)添加编译核心数。
    核模块构建:构建与内核编译类似,设置交叉编译环境后使用make即可编译。

测试机:运行主机是LoongArch64结构,使用龙芯的3A5000CPU,配备7A2000桥片。桥片上集成了LG110显卡,该显卡所使用的固件需要在
% bocreate 缓冲重用
% pb_cache_bucket = gsgpu_get_heap_index(domain, flags);


\section{功能测试}

\section{性能测试}

% \section{启动时间优化}

% \section{PAGE\_SIZE调整带来的影响评估}

% \section{性能分析}

% \section{兼容性测试}

% \subsection{surfaceflinger关键流程分析}

% \subsection{性能分析软硬件环境和方法} gtestopengl 测试套件

% \subsection{结论}

% \section{???}

