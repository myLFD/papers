% !TeX root = ../main.tex

\chapter{总结与展望}

\section{工作总结}

本文基于安卓和龙芯GPU,设计并实现了一个安卓图形栈,填补了LoongArch架构下安卓图形系统的空缺,可以运行在龙芯常见的硬件环境下。由于GPU的软硬件环境较为复杂,本文仅涉及
部分关键模块及库的构建。主要工作包括:

1.内核驱动设计与多版本适配:基于Linux DRM框架,设计了17类IOCTL命令码,覆盖显存管理(GEM)、命令流控制(CS)、虚拟地址映射(VA)等核心功能。
针对Linux内核版本差异(4.19至5.10),提出动态编译检测方案,通过条件编译实现接口兼容性,解决了闭源驱动的多版本适配问题。

2.硬件抽象层实现和镜像定制:图形缓存分配器基于DMA-BUF框架,采用缓冲区复用与零拷贝技术,支持显存与系统内存的动态分配策略,优化了图形缓冲区的分配效率。
通过HIDL接口实现绑定式服务,确保跨进程共享的安全性。
硬件混合渲染器结合DRM/KMS子系统,实现图层合成与原子提交功能,支持主平面与光标平面的硬件加速合成,并通过VSync同步机制降低显示延迟。
基于Mesa框架实现龙芯OpenGL ES 2.0/3.0支持,集成龙芯显卡后端。
针对龙芯GPU的16KB页表特性,定制Android内核与Bionic库,修改文件系统与内存管理模块,成功适配Android 12的页表兼容性要求。

3.系统验证与测试:
通过Android官方测试集对图形栈各模块进行验证,内核驱动通过环形缓冲区测试(IB、GFX、XDMA)及21项libdrm功能测试;
Gralloc模块通过180项GrallocTypes\_test用例,覆盖元数据序列化、HDR支持等场景;
SurfaceFlinger服务通过898项测试,包括多显示器管理、事务处理及渲染性能验证。
实验结果表明,系统首次在LoongArch平台上实现Android图形栈的完整运行。

本文首次在龙芯平台上实现Android图形系统的全栈适配,基于龙芯显卡实现了gralloc与HWC模块,并构建OpenGL ES等支持,实现了龙芯LG显卡与安卓框架的高效协同,
为后续基于该平台的开发和研究奠定了技术基础。

\section{未来展望}

GPU所涉及的软件支持很复杂,包括编译器后端生成、图形编程接口的支持等,本文只涉及部分模块和内容。同样安卓系统也作为一个即其庞大而复杂的系统,其图形系统更是作为最复杂的系统之一,
从工程实现和调试的角度来看都有不小的难度。本文所做的工作仅涉及部分模块的实现以及构建,接下来工作可以分成几个方向:

1.系统及硬件升级:当前方案仅适配Android S版本与龙芯3A5000平台,并且为了强行支持龙芯硬件特性对安卓系统镜像进行了许多定制,导致ART运行以及zygote等安卓APP无法运行,
在测试上也带来了一定的困难,未来需扩展至更高版本安卓(如Android 15)支持更多的硬件特性及新一代龙芯硬件(如LG200),并针对Vulkan API支持等场景进行性能优化。
从而实现在实际场景下的验证,如3D游戏、视频渲下的性能评估,运行一些常见的JAVA性能测试工具,测试帧率、功耗等关键指标。
而LG110作为一个过渡性、研究性的GPU,是龙芯制造国产GPU道路上一个尝试性的产品,其本身具有许多瑕疵,不适合继续深入做研究,应尽快过渡至LG200。

2.该方案目前的重点是搭建一个可运行的图形环境,并探索相应的软件依赖方案,其实现上仍有许多不足。在完成对功能的完善后可以考虑更多定制化的方案进行优化,以实现更高效的图形环境。

