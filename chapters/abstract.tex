% !TeX root = ../main.tex

\ustcsetup{
  keywords = {
    图形栈 ,龙芯GPU ,图形驱动 ,Android ,SurfaceFlinger
  },
  keywords* = {
    Graphic Stack ,Loongson GPU,GPU Driver,Android ,SurfaceFlinger
  },
}
\begin{abstract}

  随着信息技术的快速发展,移动端图形处理需求日益增长,Android 作为全球主流移动操作系统,其图形系统的性能与兼容性直接影响用户体验。
  然而,当前 Android 生态主要围绕 ARM Mali、Qualcomm Adreno 等 GPU 进行优化,对国产 GPU 的支持仍处于初步探索阶段。
  与此同时,LoongArch 作为自主研发的指令集架构,正逐步构建完整的软件生态,但在 Android 平台上的图形系统仍面临适配挑战。
  针对这一现状,本文开展了对 LoongArch 架构下 GSGPU 显卡的适配研究,填补了该架构在 Android 生态中的技术空白。

  本文基于 LoongArch 处理器与 GSGPU 显卡,围绕 Android 图形系统进行了以下工作:
  
  1.设计并实现了显卡驱动内核模块,定义了 DRM 内核模块与用户态之间的交互命令码,实现了从用户空间提交 GPU 命令到 GPU 处理器的高效传输。

  2.基于龙芯硬件接口,实现了 Android 关键依赖模块——硬件混合渲染器(HWC)和图形缓存分配器(Gralloc)。其中,图形缓存分配器采用缓冲区复用与零拷贝技术,优化缓冲区分配与映射效率;
  硬件混合渲染器则作为合成器,为上层 Android 应用提供帧合成与显示支持。

  3.解决了龙芯平台 Android 系统的软件栈适配问题,增加 OpenGL ES 支持,并针对硬件特性定制了部分 Android 系统镜像,成功打通系统显示通路,构建了完整的支持龙芯 GPU 的 Android 运行环境。

  为验证系统的正确性与可用性,本文基于 Android 官方测试集对各模块进行了功能测试。
  实验结果表明,本文首次在 LoongArch 平台上成功适配 GSGPU 显卡,并构建了完整的 Android 图形系统,确保了国产 GPU 在 Android 生态中的可用性。
  研究成果实现了国产 GPU 与 Android 框架的高效协同,为后续 LoongArch 平台的国产化替代与生态完善奠定了坚实的技术基础。
  
\end{abstract}

\begin{abstract*}
  This is a sample document of USTC thesis \LaTeX{} template for bachelor,
  master and doctor. The template is created by zepinglee and seisman, which
  orignate from the template created by ywg. The template meets the
  equirements of USTC thesis writing standards.

  This document will show the usage of basic commands provided by \LaTeX{} and
  some features provided by the template. For more information, please refer to
  the template document ustcthesis.pdf.
\end{abstract*}
