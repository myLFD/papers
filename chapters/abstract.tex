% !TeX root = ../main.tex

\ustcsetup{
  keywords = {
    中国科学技术大学, 学位论文, \LaTeX{} 模板, 学士, 硕士, 博士
  },
  keywords* = {
    University of Science and Technology of China (USTC), Thesis,
    \LaTeX{} Template, Bachelor, Master, PhD
  },
}

\begin{abstract}
  移动互联网的兴起极大的促进了人类社会的发展,便捷的基础设施成了现代社会的必不可少的因素。得益于市场对于各种终端设
  备的广阔需求,国产厂商也在这场信息化浪潮中占有一席之地。当然终端设备的飞速发展离不开操作系统的支持,其中 Google 
  公司的 Android 系统由于其开源、开放的特性而广受硬件厂商的青睐。其中图形显示模块作为安卓系统中最复杂且重要的几个模块之一,
  其性能和表现与用户体验息息相关,将安卓这一全球主流操作系统与龙芯硬件平台结合,不仅可以丰富完善龙芯的软件生态环境,还可以
  探索移动端的国产自主芯片替代,提高龙芯在开源操作系统社区的影响力和竞争力。
  本文以龙芯自研显卡芯片以及Android S版本为基础,完成了以下几个方面的工作:

  % 1、安卓基础编译系统的构建,基础依赖库bionic的移植与测试
  
  % 2、图形栈依赖组件的移植

  % 3、???

  % 本文以龙芯的自研GPU为基础,完成了以下及各方面的工作
  % 1、绪论
  %  研究背景和意义
  %  国内外研究现状
  %  本文主要工作
  %  本文的组织架构
  % 2、前置相关技术支持
  %   soong
  %   bionic
  %   kernel-common
  % 3、HAL模块设计与实现
  %   组件实现
  %     HAL
  %     gralloc
  %     hwcomposer
  %     mesa
  %     libdrm
  %   page size调整
  % 4、性能分析和优化(安全性和稳定性分析)
  %   启动时间优化
  %   硬件组合渲染器
  %   轻量级gbm实现minigbm
  % 5、测试与验证
  % 6、工作与展望
\end{abstract}


% 1. 龙芯平台的 Android 显示模块架构设计
% 内容概要:详细介绍龙芯平台的硬件架构及其与 Android 系统的兼容性,分析设计 HAL 层的必要性和架构选择。
% 研究重点:讨论 HAL 层的模块化设计、接口定义及其与 Android 框架的交互。
% 2. Android HAL 层的性能优化
% 内容概要:探讨如何针对龙芯平台的特性优化 Android 显示模块的 HAL 层,提升图形渲染效率。
% 研究重点:分析性能瓶颈,提出优化策略,如缓存机制、异步渲染等,比较优化前后的性能数据。
% 3. 基于龙芯的 Android 显示模块与其他平台的比较
% 内容概要:将龙芯平台的 Android 显示模块与其他主流平台(如 ARM、x86)进行比较,分析各自的优势与劣势。
% 研究重点:从性能、兼容性、能效等多个维度进行比较,提供实证数据支持。
% 4. 龙芯平台显示模块的应用场景与案例研究
% 内容概要:分析龙芯平台的 Android 显示模块在实际应用中的表现,如工业控制、嵌入式系统等。
% 研究重点:通过具体的案例研究,展示 HAL 层的实际应用效果和用户反馈,探讨其在特定场景下的适用性。
% 5. 安全性与稳定性分析:龙芯平台的 HAL 层实现
% 内容概要:研究龙芯平台 Android 显示模块 HAL 层的安全性和稳定性,探讨潜在的安全风险及其解决方案。
% 研究重点:分析 HAL 层的安全架构,讨论如何防范攻击、保护用户数据及提升系统稳定性。
% 6. 基于龙芯的 Android 显示模块的标准化与可扩展性
% 内容概要:探讨如何在龙芯平台上实现 Android HAL 层的标准化,以便于未来的扩展和维护。
% 研究重点:分析现有标准(如 OpenGL ES、Vulkan)在 HAL 层中的应用,提出如何支持新兴技术的建议。
% 7. 用户体验与界面设计:基于龙芯的 Android 显示模块
% 内容概要:研究基于龙芯平台的 Android 显示模块对用户体验的影响,特别是在界面响应和渲染效果方面。
% 研究重点:通过用户研究和测试,评估不同设计决策对用户体验的影响,提出改进建议。
% 8. 跨平台适配:在龙芯上实现 Android HAL 层的挑战与解决方案
% 内容概要:分析在龙芯平台上实现 Android 显示模块 HAL 层时所面临的跨平台适配挑战。
% 研究重点:探讨如何解决兼容性问题,保证在不同硬件和软件环境下的稳定性和性能。
% 这些方向可以为您的研究提供多种切入点,选择一个或多个方向深入探讨,可以形成一篇有深度的学术论文。

% 详细说说第一个,显示模块架构设计


% GPT-4o-Mini

% Poe
% 论文方向:龙芯平台的 Android 显示模块架构设计
% 1. 引言
% 背景介绍:
% 随着移动设备和嵌入式系统的普及,Android 操作系统在不同硬件平台上的应用越来越广泛。龙芯作为中国自主研发的 CPU 平台,其在嵌入式和桌面应用中的潜力逐渐显现。
% 研究目的:
% 本文旨在探讨基于龙芯平台的 Android 显示模块的架构设计,分析其与 Android 系统的兼容性以及实现 HAL 层的必要性。
% 2. 龙芯平台概述
% 硬件架构:
% 介绍龙芯 CPU 的基本架构,包括处理器核心、内存管理、图形处理单元(GPU)等。
% 讨论龙芯平台的优势,如开放性、可定制性和能效比等。
% 与 Android 系统的兼容性:
% 分析龙芯平台在支持 Android 应用方面的能力,包括对 Android 运行时(ART)和 Linux 内核的兼容性。
% 3. Android 显示模块的 HAL 层设计
% HAL 层的概念:
% 解释 HAL 层的作用,如何为上层应用和服务提供硬件接口抽象,促进软件与硬件的解耦。
% 模块化设计:
% 讨论显示模块的模块化设计理念,如何将功能划分为不同的组件,如:
% Framebuffer HAL:处理帧缓冲区的管理和操作。
% Graphic Composer HAL:负责图形合成和显示。
% Display HAL:管理显示设备的初始化和控制。
% 接口定义:
% 描述 HAL 接口的设计,使用 C/C++ 语言定义模块功能,包括支持的函数和数据结构。
% 提供具体的接口示例,如 init(), setDisplayMode(), flush() 等方法。
% 4. 显示模块的实现
% 模块实现细节:
% 介绍如何实现 HAL 层的各个组件,详细描述每个模块的功能和实现方法。
% 使用龙芯特定的硬件访问方式,如 MMIO(内存映射输入输出)等。
% 性能优化:
% 讨论在实现过程中如何进行性能优化,包括减少上下文切换、使用硬件加速等方法。
% 5. 与 Android 框架的交互
% JNI 绑定:
% 解释如何通过 JNI 将 HAL 层与 Android 框架连接,使 Java 层能够调用底层 C/C++ 实现。
% 数据流和事件处理:
% 描述数据流在 HAL 层和 Android 框架之间的传递过程,如何处理输入事件和显示更新。
% 6. 测试与验证
% 测试策略:
% 制定测试方案,验证 HAL 层的功能性和稳定性,确保与 Android 系统的良好兼容性。
% 性能评估:
% 使用基准测试工具,评估显示模块的性能,比较不同优化策略下的效果。
% 7. 结论与未来工作
% 总结:
% 总结基于龙芯平台的 Android 显示模块的架构设计的主要贡献与创新点,强调其在中国自主技术发展中的意义。
% 未来工作:
% 提出对未来工作的展望,如进一步的优化策略、对新硬件的适配以及与其他 Android 功能的集成。
% 参考文献
% 列出相关文献,包括 Android HAL 设计文档、龙芯技术文档、以及相关的学术论文和书籍。


%\begin{abstract}
%  摘要分中文和英文两种,中文在前,英文在后,博士论文中文摘要一般 800~1500 个汉字,硕士论文中文摘要一般 500~1000 个汉字。
%  英文摘要的篇幅参照中文摘要。
%
%  关键词另起一行并隔行排列于摘要下方,左顶格,中文关键词间空一字或用分号“,”隔开,英文关键词之间用逗号“,”或分号“;”隔开。
%
%  中文摘要是论文内容的总结概括,应简要说明论文的研究目的、基本研究内容、研究方法或过程、结果和结论,突出论文的创新之处。
%  摘要应具有独立性和自明性,即不用阅读全文,就能获得论文必要的信息。
%  摘要中不宜使用公式、图表,不引用文献。
%
%  中文关键词是为了文献标引工作从论文中选取出来用以表示全文主题内容信息的单词和术语,一般 3~8 个词,要求能够准确概括论文的核心内容。
%\end{abstract}

\begin{abstract*}
  This is a sample document of USTC thesis \LaTeX{} template for bachelor,
  master and doctor. The template is created by zepinglee and seisman, which
  orignate from the template created by ywg. The template meets the
  equirements of USTC thesis writing standards.

  This document will show the usage of basic commands provided by \LaTeX{} and
  some features provided by the template. For more information, please refer to
  the template document ustcthesis.pdf.
\end{abstract*}
