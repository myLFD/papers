% !TeX root = ../main.tex

\ustcsetup{
  keywords = {
    图形栈,龙芯GPU,图形驱动,安卓,表面合成器
  },
  keywords* = {
    Graphic Stack,Loongson GPU,GPU Driver,Android,SurfaceFlinger
  },
}
\begin{abstract}
  随着信息技术的范式变革,移动端图形处理需求呈现指数级增长。作为占据全球市场份额76\%的移动操作系统,
  Android凭借其图形子系统的性能与兼容性,已成为影响用户体验的关键要素。其中,GPU驱动作为协调硬件算力释放与系统稳定性的核心组件,
  直接决定着图形渲染管线的效能边界。然而在Android商业化生态中,主流GPU厂商的技术壁垒导致硬件规格文档与驱动源码获取异常困难,
  这严重制约了基于自主架构的图形系统研究。针对这一现状,本研究通过实现龙芯GPU在LoongArch指令集架构下对Android S系统的完整适配,
  填补了自主指令集在移动计算生态中的关键技术空白,具有重要的研究价值和应用意义。本文基于LoongArch架构与龙芯显卡,围绕 Android 图形系统进行了以下工作:
  
  1.设计并实现了显卡驱动内核模块,设计了 DRM 内核模块与用户态之间的交互命令码,实现了从用户空间提交 GPU 命令到 GPU 处理器的高效传输。

  2.基于龙芯硬件接口,实现了 Android 关键依赖模块——硬件混合渲染器(HWC)和图形缓存分配器(Gralloc)。
  其中,图形缓存分配器采用缓冲区复用与零拷贝技术,优化缓冲区分配与映射效率;
  硬件混合渲染器则作为合成器,为上层 Android 应用提供帧合成与显示支持。

  3.解决了龙芯平台 Android 系统的软件栈适配问题,增加 OpenGL ES 支持,并针对硬件特性定制了部分 Android 系统镜像,
  成功打通系统显示通路,构建了完整的支持龙芯 GPU 的 Android 运行环境。

  为验证系统的正确性与可用性,本文基于 Android 官方测试集对各模块进行了功能性测试。
  结果表明,本文率先在 LoongArch 平台上成功构建并稳定运行了支持龙芯 GPU 的 Android 图形系统。
  该研究实现了龙芯 LG 系列显卡与 Android 框架的高效协同,为后续 LoongArch 平台的 Android 系统部署与生态完善奠定了坚实基础。
  
\end{abstract}

\begin{abstract*}
  With the paradigm shift in information technology, the demand for mobile graphics processing has been growing exponentially. As a mobile operating system occupying 76\% of the global market share,
  Android, by virtue of the performance and compatibility of its graphics subsystem, 
  has become a key factor influencing user experience. Among these, the GPU driver, 
  serving as the core component that coordinates hardware computing power release and system stability,
  directly determines the performance boundary of the graphics rendering pipeline. 
  However, in the commercial Android ecosystem, technical barriers set by mainstream GPU manufacturers 
  have made it extremely difficult to obtain hardware specification documents and driver source code,
  which severely constrains research on graphics systems based on independent architectures. 
  In response to this situation, this study achieves the complete adaptation of the Loongson GPU under 
  the LoongArch instruction set architecture to the Android S system
  , filling a critical technical gap for domestic instruction sets in the mobile computing ecosystem and 
  holding significant research value and practical importance. Centered on the Android graphics system and 
  based on the LoongArch architecture and Loongson GPU, this paper covers the following work:
  
  1.Designing and implementing the graphics driver kernel module, specifying the interaction command codes between 
  the DRM kernel module and user space, thereby enabling the efficient transmission of GPU commands from 
  user space to the GPU processor.
  
  2.Based on the Loongson hardware interface, implementing the essential Android modules—namely the Hardware Composer (HWC) and the Graphics Buffer Allocator (Gralloc).
  Among these, the graphics buffer allocator utilizes buffer reuse and zero-copy techniques to optimize buffer allocation and mapping efficiency;
  the hardware composer serves as a compositor, providing frame composition and display support for upper-level Android applications.
  
  3.Resolving the software stack adaptation issue of the Loongson platform’s Android system, adding OpenGL ES support, and customizing part of the Android system image according to hardware features,
  successfully establishing the system display pathway and creating a complete Android runtime environment supporting the Loongson GPU.
  
  To verify the correctness and usability of the system, this paper conducts functionality tests on each module based on the official Android test suite.
  The results indicate that this study has taken the lead in successfully constructing and stably operating an Android graphics system supporting the Loongson GPU on the LoongArch platform.
  This achievement has enabled efficient collaboration between Loongson LG-series GPUs and the Android framework, laying a solid foundation for subsequent deployment of Android systems and ecosystem enhancement on the LoongArch platform.
  
\end{abstract*}
